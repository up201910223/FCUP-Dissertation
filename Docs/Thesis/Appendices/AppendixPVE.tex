
\chapter*{proxmox\_api endpoints} % Main appendix title

\label{pve_api_appendix} % Change X to a consecutive letter; for referencing this appendix elsewhere, use \ref{AppendixX}

This library is responsible for interacting with the Proxmox VE API, automating management tasks such as starting, stopping, and templating virtual machines and containers.  

Unless otherwise specified, all methods return a boolean value. In the event of network errors:
\begin{itemize}
  \item Functions expecting a boolean will return \texttt{False}.
  \item Functions expecting other types will return \texttt{None}.
\end{itemize}

All other exceptions are propagated and should be caught by the caller.

\subsection*{Module: \texttt{proxmox\_vm\_actions}}

\begin{verbatim}
_get_status(proxmox_host, session, vm_id)
    Queries the state of a VM. For internal use. Returns the raw response.

acheck_free_id(proxmox_host, session, id)
    Checks if given VM/CT ID is unused. Does not reserve it.

create(proxmox_host, session, template_id, clone_id, hostnames)
    Clones the specified template VM with the given hostname.

check_vm_status(proxmox_host, session, vm_id)
    Checks if the VM is running and the QEMU guest-agent is active.

check_vm_is_template(proxmox_host, session, vm_id)
    Checks if a VM is in template format.

start(proxmox_host, session, vm_id)
    Starts the specified VM.

stop(proxmox_host, session, vm_id)
    Stops the specified VM.

template(proxmox_host, session, vm_id)
    Transforms the VM into a template.

destroy(proxmox_host, session, vm_id)
    Destroys the specified VM.
\end{verbatim}

The \texttt{session} parameter must be created using the \texttt{connection} module in \texttt{utils/}, and must contain valid credentials for the Proxmox cluster.

\subsection*{Module: \texttt{proxmox\_vm\_firewall}}

\begin{verbatim}
create_proxmox_vm_isolation_rules(proxmox_host, first_vm_id, last_vm_id, allowed_vm_ip, session)
    Enables firewall and blocks communication between student VMs.
    Only allows communication with the allowed VM (typically the teacher VM).

delete_proxmox_vm_isolation_rules(proxmox_host, first_vm_id, last_vm_id, allowed_vm_ip, session)
    Removes previously set firewall rules, re-enabling inter-VM communication.
\end{verbatim}

\subsection*{Module: \texttt{utils}}

Various utility functions to support Proxmox interaction:
\begin{itemize}
  \item \texttt{connection.proxmox\_connect}:  
    Authenticates and returns an HTTP session with a valid authentication cookie.  
    See the ProxmoxVE authentication documentation for details.
    
  \item \texttt{proxmox\_base\_uri\_generator}:  
    Generates the base URI for API calls to Proxmox.

  \item \texttt{proxmox\_vm\_ip\_fetcher}:  
    Retrieves the current IP address or hostname of a VM or container given its ID.
\end{itemize}