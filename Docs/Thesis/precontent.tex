
%-------------------------------------------------------------------------
%	QUOTATION PAGE
%-------------------------------------------------------------------------
%\quotepage{Matt Smith as \emph{The Doctor}, written by Matthew Graham}
%{
%	I am and always will be the optimist, the hoper of far-flung hopes and the
%	dreamer of \newline improbable dreams
%}

%-------------------------------------------------------------------------
%	DEDICATORY
%-------------------------------------------------------------------------

%\begin{dedicatory}
%	Dedicated to (optional) 
%\end{dedicatory}

%-------------------------------------------------------------------------
%	ACKNOWLEDGEMENTS PAGE
%-------------------------------------------------------------------------
\addtocontents{toc}{\protect\setcounter{tocdepth}{-1}}
\begin{acknowledgements}

Firstly i would like to thank my friends and family for encouraging me to pursue both my Bachelor's and Master's degrees. 
Their love and support were fundamental in this endeavor.

Additionally i want to express my sincerest gratitude to my girlfriend Daniela for all the love, support, company, all 
the time spent playing board games and just being herself, as well as her family for all the support that they give me.

Furthermore i would like to thank my Master's supervisor Professor Rui Prior for allowing me to explore this topic, as well 
as all the ideas and feedback he gave me during development.

Likewise i would like to thank Instituto de Telecomunicações (UIDB/50008/2020) for hosting this work.

Lastly i would like to thank Ginjas, for just being him.

Thank you.

\vspace*{10cm}
\newfontfamily\sinhalaFont{Noto Sans Sinhala}
\tikz[baseline]{
  \node[opacity=0.05, text=black, font=\sinhalaFont] at (0,0) {ඞ};
}

\end{acknowledgements}
\addtocontents{toc}{\protect\setcounter{tocdepth}{3}}
%\addvspacetoc{0.3cm} % Add a gap in the Contents, for aesthetics


%-------------------------------------------------------------------------
%	ABSTRACT PAGE (PORTUGUESE)
%-------------------------------------------------------------------------
\addtocontents{toc}{\protect\setcounter{tocdepth}{-1}}
\begin{abstract}[
	thesistitle={Sistema para Avaliações Práticas de Administração de Redes },
	title={Resumo},
	degree={Mestrado em Engenharia de Redes e Sistemas Informáticos},
	nameconnector={por},
        keywordsname={Palavras-chave},
        keywords={ Proxmox VE, GNS3, automação, Nornir, FastAPI, assíncrono, comunicação, emulação, ferramenta de ensino}]
\begin{otherlanguage}{portuguese}

Os métodos convencionais de formação em rede de computadores, que dependem fortemente do hardware físico, são muitas vezes proibitivamente 
caros e não são escaláveis. As tecnologias de virtualização e emulação resolvem esta limitação, permitindo a execução 
simultânea de vários dispositivos simulados numa única máquina. No entanto, a avaliação da correção das configurações dos 
dispositivos de rede continua a ser um processo manual, propenso a erros, em que configurações incorrectas podem conduzir 
a falhas críticas em toda a rede.

A primeira fase deste projeto, conduzida por um estudante anterior, centrou-se na investigação da viabilidade de construir um sistema deste 
tipo e na produção de um conjunto de componentes soltos. Este trabalho representa a segunda fase, focando-se no desenho e implementação de um 
sistema para avaliação automática de topologias de rede, com suporte para um ambiente de dispositivos heterogéneos, no seguimento direto de um 
trabalho anterior. O sistema terá como objetivo fornecer um ambiente de trabalho completo para os alunos, uma interface Web para que tanto os 
alunos como os professores possam interagir com a plataforma e capacidades para automatizar a avaliação das configurações feitas pelos 
alunos, permitindo que os professores se concentrem no ensino e na orientação dos alunos, em vez de terem de efetuar 
verificações manuais repetitivas dos trabalhos dos alunos.

Os desenvolvimentos deste projeto incluem a extensão e integração do trabalho anterior, bem como o desenvolvimento de uma 
aplicação web utilizando FastAPI para alavancar as capacidades de comunicação assíncrona que se revelaram essenciais para 
garantir que o sistema pode escalar.


\end{otherlanguage}
\end{abstract}
\addtocontents{toc}{\protect\setcounter{tocdepth}{3}}
%-------------------------------------------------------------------------
%	ABSTRACT PAGE
%-------------------------------------------------------------------------
\addtocontents{toc}{\protect\setcounter{tocdepth}{-1}}
\begin{abstract} 

Conventional computer network teaching methods, which depend heavily on physical hardware, are often prohibitively expensive and lack scalability. 
Virtualization and emulation technologies address this limitation by allowing multiple simulated devices to run concurrently 
on a single system. Nevertheless, assessing the correctness of network device configurations continues to be a manual, 
error-prone process, where misconfigurations can lead to critical failures across the network.

The first phase of this project, conducted by a previous student, focused on investigating the viability of building such a system and 
producing a set of loose components. This work represents the second phase,centered on designing and implementing a system for automated assessment 
of network topologies, designed to support a diverse range of network devices. Building upon previous research, the system aims to deliver a complete, 
web-based working environment where both students and instructors can interact with the platform. Its key feature is the automation of configuration 
assessments made by students, which reduces the manual workload for instructors, allowing them to focus more on teaching and guiding 
rather than repetitive checking of assignments.

The developments of this project include the extension and integration of previous work as well as the development of a web 
application using FastAPI to leverage asynchronous communication capabilities which proved essential to ensure the system can 
scale.


\end{abstract}
\addtocontents{toc}{\protect\setcounter{tocdepth}{3}}

%-------------------------------------------------------------------------
%	LIST OF CONTENTS/FIGURES/TABLES
%-------------------------------------------------------------------------

\addtocontents{toc}{\protect\setcounter{tocdepth}{-1}}

\tableofcontents % Write out the Table of Contents

\addtocontents{toc}{\protect\setcounter{tocdepth}{3}}
\addvspacetoc{0.3cm}

%\listoftables % Write out the List of Tables

\listoffigures % Write out the List of Figures



%\addvspacetoc{0.3cm}

%-------------------------------------------------------------------------
%	PHYSICAL CONSTANTS/OTHER DEFINITIONS
%-------------------------------------------------------------------------

%\begin{listofcontants}
%	\const{My little ponny test of magical rainbow}{$mn/mp$}
%    {$2.997\ 924\ 58\times10^{8}\ \mbox{ms}^{-\mbox{s}}$}
%   \const{Vaccuum permeability test of magical rainbow for a specific case of
%   condensed matter physics}
%   {$\epsilon_0$}{$2.997\ 924\ 58\times10^{8}\ \mbox{ms}^{-\mbox{s}}$}
%	\const{Speed of Light test of magical rainbow}{$c$}
%    {$2.997\ 924\ 58\times10^{8}\ \mbox{ms}^{-\mbox{s}}$}
%\end{listofcontants}


%-------------------------------------------------------------------------
%	SYMBOLS
%-------------------------------------------------------------------------

%\begin{listofsymbols}
%	\symb{$F_{\mu\nu}$}{Maxwell tensor}{F}
%	\symb{$a$}{distance}{m}
%	\\
%	\symb{$\omega$}{angular frequency}{rads$^{-1}$}
%\end{listofsymbols}


%-------------------------------------------------------------------------
%	NOTATION
%-------------------------------------------------------------------------

% \newcommand\notationname{Notation and Conventions}
% \addtotoc{\notationname}
% \fancyhead[LO]{\textsc{\notationname}}

% \input{Notation}



%-------------------------------------------------------------------------
%	ABBREVIATIONS
%-------------------------------------------------------------------------

\newacronym{cs}{CS}{Computer Science}

\newacronym{dcc}{DCC}{FCUP-Department of Computer Science}

\newacronym{gns3}{GNS3}{Graphical Network Simulator-3}

\newacronym{rest}{REST}{Representational State Transfer}

\newacronym{api}{API}{Application Programming Interface}

\newacronym{qemu}{QEMU}{Quick Emulator}

\newacronym{iou}{IOU}{IOS on Unix}

\newacronym{ios}{IOS}{Internetworking Operating System}

\newacronym{vpcs}{VPCS}{Virtual PC Simulator}

\newacronym{vm}{VM}{Virtual Machine}

\newacronym{http}{HTTP}{Hypertext Transfer Protocol}

\newacronym{json}{JSON}{JavaScript Object Notation}

\newacronym{wsgi}{WSGI}{Web Server Gateway Interface}

\newacronym{asgi}{ASGI}{Asynchronous Server Gateway Interface}

\newacronym{oas}{OAS}{OpenAPI Specification}

\newacronym{pve}{Proxmox VE}{Proxmox Virtual Environment}

\newacronym{kvm}{KVM}{Kernel-based Virtual Machine}

\newacronym{lxc}{LXC}{Linux Containers}

\newacronym{jwt}{JWT}{JSON Web Token}

\newacronym{ldap}{LDAP}{Lightweight Directory Access Protocol}

\newacronym{ssh}{SSH}{Secure Shell}

\newacronym{lvm}{LVM}{Logical Volume Manager}

\newacronym{lvmt}{LVM-Thin}{LVM Thin Provisioning}

\newacronym{cow}{CoW}{Copy-on-Write}

\newacronym{orm}{ORM}{Object-Relational Mapping}

\newacronym{spice}{SPICE}{Simple Protocol for Independent Computing Environments}

\newacronym{dry}{DRY}{Don't Repeat Yourself}

\newacronym{uuid}{UUID}{Universally Unique Identifier}

\newacronym{ui}{UI}{User Interface}

\newacronym{gui}{GUI}{Graphical User Interface}

\newacronym{cpu}{CPU}{Central Processing Unit}

\newacronym{virl}{VIRL}{Virtual Internet Routing Lab}

\newacronym{vnc}{VNC}{Virtual Network Computing}

\newacronym{yaml}{YAML}{YAML Ain't Markup Language}

\newacronym{ssd}{SSD}{Solid State Drive}



\printglossary[type=\acronymtype,title={List of Abbreviations}]
