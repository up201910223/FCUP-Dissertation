% Chapter Template

% Main chapter title
%\chapter[toc version]{doc version}
\chapter{Introduction}

% Short version of the title for the header
%\chaptermark{version for header}

% Chapter Label
% For referencing this chapter elsewhere, use \ref{ChapterTemplate}
\label{Chapter1Introduction}

% Write text in here
% Use \subsection and \subsubsection to organize text

Network administration remains a fundamental discipline within\ac{cs}, given the central role that networked systems play in the functioning, 
scalability, and security of modern digital infrastructures.

Modern organizations require professionals who can design, configure, and troubleshoot increasingly complex network environments. 
Effective education must therefore bridge theoretical knowledge with practical implementation, particularly through hands-on learning 
that simulates real-world scenarios. Yet current training and assessment methods fail to meet these needs at scale, creating a growing gap 
between academic preparation and professional requirements.

This project aims to build an initial version of a system capable of assisting training in the topic of network administration, building upon previous 
work of \citet{santos2024} which solved a series of individual problems relevant to this topic, without building a system out of them.

\section{Need for Automated Assessment}

  Practical assessments are essential to prepare students for real-world challenges, enabling them to apply 
  theoretical knowledge and develop hands-on skills. However, traditional assessment methods face significant limitations.

  Creating a physical network environment for practical assessments is sure to be costly and challenging to scale for 
  large student populations. 
  While emulation and virtualization technologies offer cost-effective alternatives for creating flexible practice environments, 
  they lack built-in automated assessment capabilities. 
  Instructors must manually review each students' network assignment configuration—a process that is:

  \begin{itemize}
    \item \textbf{Time-consuming}: Manual checks grow linearly with class size.
    \item \textbf{Error-prone}: Human reviewers may overlook misconfigurations.
    \item \textbf{Risk of inconsistency}: Automated assessment ensures fully deterministic and consistent evaluation.
  \end{itemize}

  Automating assessments would reduce instructor workload, freeing time for student support as well as ensure consistent, objective grading 
  and simultaneously enable immediate feedback for learners.

  Previous work by \citet{santos2024} has shown that it is possible to automate the validation of basic network commands, such as verifying 
  the output of a ping command between a specific machine and a target IP address. This demonstrates that connectivity checks 
  can be automated, paving the way for network automated assessment.

  \subsection{Limitations of Current Solutions}

    Existing approaches to network education suffer from two critical gaps:

    \begin{enumerate}
        \item \textbf{Single-vendor focus}: Most tools (e.g., Cisco's Packet Tracer) are designed for specific vendor ecosystems, failing 
        to prepare students for heterogeneous real-world networks where multi-vendor interoperability is essential.
        \item \textbf{Missing assessment component}: Some education institutions' curricula, such as our case in\ac{dcc}, forgo practical 
        assessments entirely, meaning students complete networking exercises without getting feedback or validation. This lack of feedback 
        can lead to students having gaps in their knowledge, hindering their performance when they join the workforce, as they may not have 
        acquired sufficient practical skills in the identification and solving of common problems.
    \end{enumerate}

\section{Aims and Objectives}

  Designing and implementing a system that can automatically evaluate network administration assignments is one of the aims 
  of this work. It will be utilised as a tool in classes and tests, allowing instructors to spend more time directly helping 
  and guiding students.

  One of the objectives needed to accomplish this, is to have such a system be able to perform assessments whenever requested 
  by a student, in a fully automated manner.
  To have a wide range of support for vendors and device types is also highly desirable as exposing students to as much 
  variety as possible will better prepare them for more heterogeneous environments of the real world.
  It will also be important to have assessment for configurations as well as connectivity.

  Another aim of this project is to create a modular and cohesive back-end system out of the components created by 
  \citet{santos2024}.
  Achieving this will require further development of the components as well as creating capabilities to communicate between 
  them.

  

\section{Organization}

  Besides the Introduction, this work includes 6 more chapters:

  \begin{enumerate}
      \item \textbf{Chapter 2} presents an overview of the system components and underlying technologies.
      \item \textbf{Chapter 3} reviews related work, highlighting lessons learned and how this project differentiates itself from existing solutions.
      \item \textbf{Chapter 4} provides a detailed description of the system architecture and its layers. It also outlines the project's functional use cases.
      \item \textbf{Chapter 5} focuses on implementation details, emphasizing design decisions and their realization.
      \item \textbf{Chapter 6} discusses performance evaluation results, demonstrating the scalability of solutions introduced in this work.
      \item \textbf{Chapter 7} gives an overview of met aims and objectives, reviews lessons learned throughout, and goes 
      over ideas for future work.
  \end{enumerate}