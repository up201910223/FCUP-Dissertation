% Chapter Template

% Main chapter title
%\chapter[toc version]{doc version}
\chapter{Related Work}

% Short version of the title for the header
%\chaptermark{version for header}

% Chapter Label
% For referencing this chapter elsewhere, use \ref{ChapterTemplate}
\label{Chapter3RelatedWork}

% Write text in here
% Use \subsection and \subsubsection to organize text

This chapter focuses on placing our project within the context of existing solutions and related work. The primary 
goal of this project is to develop a system capable of automatically assessing network topologies by validating device 
configurations and executing tests across a virtual network.

While automated assessment systems are well established in the field of programming education, receiving student-submitted 
code and running it against predefined test cases, equivalent systems for network exercises are far less common. Tools like 
Mooshak \cite{Leal2003567} and similar platforms have proven effective for evaluating programming assignments and are widely adopted in academic 
settings.

At first glance, adapting automated assessment approaches to network topologies may seem straightforward. 
However, network assessment introduces a set of challenges like validating the configuration of multiple devices in a distributed and 
stateful environment. These devices may differ in type, vendor, or firmware version, requiring tailored communication methods and output 
validation strategies for each case.

This chapter explores existing tools, in particular, Mooshak and Packet Tracer, highlighting their capabilities, limitations, and how our  
project builds upon or diverges from them.

\section{Automated Assessment Systems}
    While not directly related, programming assessment systems are the main inspiration for this project. They are widely
    deployed in universities and other educational institutions. These systems receive, as input, code from students and 
    subsequently run tests on it, outputting a score and even being configurable to provide students the first test case that 
    they failed in, guiding students to the solution without handing it out.

    A key differentiator of the proposed system is its support for multiple valid configurations across several devices to achieve 
    a correct solution to a network exercise. In contrast, many existing automated assessment systems, particularly those used in 
    programming, assume a single expected output for a given input, which does not accommodate the variability and flexibility 
    often found in networking scenarios.

    
    One key difference is that automated assessment systems dont always provide a working environment for the 
    students to test their code, owing to the fact that students might prefer to user their own development environment for 
    initial development and testing. However setting up a networking lab can be a daunting task for students, especially when 
    they are just starting out. By providing a pre-configured environment, students can focus on learning the concepts and skills 
    they need to succeed in their studies, rather than spending time troubleshooting their setup. Another key difference is that 
    automated assessment systems dont have to interact with distributed systems or deal with matters of concurrency as is the case 
    in a network administration assignment, as to validate it, it will be necessary to interact with multiple devices, even if 
    they are virtually hosted on the same physical machine.

    \subsection{Mooshak and lessons learned}

        In our context, in the\ac{dcc}, Mooshak is commonly deployed to be used in the context of classes, exams and even 
        programming contests.

        Mooshak is a web-based system for managing programming contests and also to act as an automatic judge of programming 
        contests \cite{Leal2003567}. It supports a variety of programming languages like Java, C, etc. Under each contest, students 
        will find one more problem definitions each containing varying sets of test cases in input-output pairs. After submiting 
        their solution, the system will compile and run the code against an instructor-provided solution and compare the outputs obtained 
        for the same inputs giving a score based on the the amount of test cases passed.

        Mooshak provides a structured approach to test coding and problem solving skills. It begins by offering a problem statement 
        coupled with an optional image and an example test case, in the form of input and expected output, as can be seen in 
        Figure~\ref{fig:mooshak1}

        \begin{figure}
            \centering
            \includegraphics[width=.95\linewidth]
                {3RelatedWork/mooshak1.png}
            \caption{A screenshot of a mooshak exercise page}
        \hfill
        \label{fig:mooshak1}
        \end{figure}

        Users can submit their proposed solution by uploading a file with their code. The system then evaluates the provided solution 
        against multiple pre-defined test cases, validating the output of the submitted code against the output of a known-good code solution, 
        giving feedback in the form of a score based on the number of test cases passed. The system may also be configured to have time and/or 
        memory constraints, to ensure that temporal and spatial complexity are also taken into account.
    
        All of these, serve to provide a thorough evaluation of the student's solution, which can help guide a student to better
        their coding and problem solving skills.

        The system can also differentiate between differing types of errors, such as not giving the expected output, poorly 
        formatted output, failure to compile or even exceeding the time limits.
        Mooshak also includes some features designed to drive competition between students, like a real time leaderboard and
        the ability to have more than 100\% of the score for a given contest.

        However, the system has limitations. It relies on plain text files for test cases and performs character-by-character output 
        comparison. This strict matching can result in false negatives when a student's output is functionally correct but formatted 
        differently than expected.

\section{Cisco Packet Tracer}

    Cisco Packet Tracer is a network \textbf{simulation} tool developed by Cisco Systems, widely used in academic environments 
    to teach networking concepts and prepare students for certifications such as the Cisco Certified Network Associate (CCNA). 
    It offers a visual interface, seen in Figure~\ref{fig:packet_tracer1}, for building and simulating virtual network topologies 
    using a variety of Cisco devices, including routers, switches, and end devices.

    \begin{figure}
        \centering
        \includegraphics[width=.95\linewidth]
            {3RelatedWork/packet_tracer1.png}
        \caption{A screenshot of a Packet Tracer exercise}
    \hfill
    \label{fig:packet_tracer1}
    \end{figure}

    Packet Tracer includes an "Activity Wizard". This feature allows for the creation of assignments with automated assessment. 
    To use this, two topologies must be provided, namely a starting network, used as a starting point, and an answer network, 
    containing the intended configurations. From this a list of all the settings applied to the answer network can be toggled 
    for validation and a custom amount of points can be assigned to each validation, as can be seen in Figure~\ref{fig:packet_tracer2}. 
    
    \begin{figure}
        \centering
        \includegraphics[width=.95\linewidth]
            {3RelatedWork/packet_tracer2.png}
        \caption{A screenshot of a Packet Tracer's Activity Wizard}
    \hfill
    \label{fig:packet_tracer2}
    \end{figure}

    Apart from the settings validation, it also provides the ability to perform connectivity tests between devices by sending a 
    simple message between them.

    Additionally, through the Activity Wizard, instructions can be included via HTML or a rich text editor which students can then 
    use as a guide. After configuration is finished a file will be obtained. This file can then be used by students to obtain the 
    starting network, the necessary instructions and real-time scoring, as points will be added when each of the correct settings are 
    inserted.

    Packet Tracer also offers a separate interface for applying configurations or services on devices, allowing students to completely skip 
    the devices' native interface seen in Figure~\ref{fig:packet_tracer3}.

    This feature can be seen as valuable in very early stages of teaching, but quickly becomes counter-productive as students should 
    grow accustomed to navigating the devices' native interfaces to properly configure them, better preparing them for professional environments.

    \begin{figure}
        \centering
        \includegraphics[width=10cm]{3RelatedWork/packet_tracer3.png}
        \caption{A screenshot of a Packet Tracer's configuration}
    \hfill
    \label{fig:packet_tracer3}
    \end{figure}

    While Packet Tracer is highly accessible and effective for introducing basic networking concepts, it is a closed-source, proprietary tool 
    designed primarily for Cisco hardware simulation. Its feature set is tailored for entry-level instruction and lacks the extensibility, platform 
    flexibility, and integration capabilities required for more advanced or automated assessment workflows. For example, Packet Tracer does not support 
    running custom operating system images or integrating external automation tools via\ac{api}s, which limits its applicability in more complex or 
    open-ended environments.

    In contrast, this project aims to allow for a more realistic and extensible lab environment. The use of real operating 
    systems and support of a wide range of vendor platforms for routers and switches, as well as Linux-based\ac{vm}s 
    is highly desirable.

    Therefore, while Cisco Packet Tracer remains a valuable educational tool, the needs of this project called for a more 
    flexible and open architecture.