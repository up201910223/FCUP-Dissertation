% Chapter Template

% Main chapter title
%\chapter[toc version]{doc version}
\chapter{Conclusion \& Future Work}

% Short version of the title for the header
%\chaptermark{version for header}

% Chapter Label
% For referencing this chapter elsewhere, use \ref{ChapterTemplate}
\label{Chapter7ConclusionFutureWork}

This work explored the design and implementation of an automated network assessment system, building upon the work of 
\citet{santos2024}. The system was developed for the deployment and evaluation of networking assignments, with each 
student interacting with a dedicated\ac{gns3} environment hosted via\ac{pve}\ac{vm}s and with the intention of being used 
in both classroom and examination settings, allowing instructors to allocate more time toward guiding and supporting 
students, rather than spending it on manual evaluation.

Throughout the project, several key aims and objectives were successfully met. A modular architecture was established, 
enabling the creation, management, and evaluation of network assignments. Integration with\ac{pve} allowed for 
automated\ac{vm} lifecycle management, while\ac{gns3} provided a flexible environment for deploying practical networking 
scenarios. With assessment being done with Nornir, which enabled scalable and repeatable command execution across multiple 
devices, and a web back-end built using FastAPI that facilitated communication between all system components.

One of the central objectives was to allow students to initiate assessments on demand. This was achieved through the use of 
the developed modules, which input a command to a given virtual network device, determine the correct command depending of 
its type (e.g. Cisco router, Linux VM have different syntax for a \texttt{traceroute}). This effort was made to ensure support for a 
diverse range of device types and vendors, as exposure to heterogeneous network environments better prepares students for 
the complexity of professional settings. While the current implementations has only support for a few device types, the 
modular structure can easily be expanded upon for broader compatibility in the future. The modules developed during the 
project offer a basis for extensibility.

Another major aim of this project was also to unify and extend the components originally developed by \citet{santos2024} into 
a cohesive back-end system. This required significant development effort, particularly in defining clear interfaces between 
components and ensuring they could operate together reliably under various conditions.

Significant progress was also made in terms of asynchronous handling and\ac{api} development. The\ac{api} design was 
iteratively refined to ensure clear boundaries between responsibilities while maintaining ease of integration with other 
components.


\section{Lessons learned}
    This project highlighted several important lessons, one of them being the criticality of asynchronous I/O in system 
    involving remote\ac{api} calls. For this, Python's asynchronous capabilities, in conjunction with libraries that support 
    them, proved invaluable. Additionally, it taught us key lessons in\ac{api} design, particularly that modularity facilitates 
    maintainability and integration, especially when multiple external systems are involved (\ac{pve},\ac{gns3}). 

    Additionally, we learned that error handling, observability, and retry mechanisms should be devised and implemented from the 
    the very early steps of development in networked systems, where silent failures during provisioning can propagate and lead to 
    complex, hard-to-diagnose issues.

    One more important lesson is the need to run tests periodically to ensure good system performance, as we found during 
    our tests that our performance was decreasing with no major errors to be seen anywhere, which highlights the importance of 
    monitoring and benchmarking.


\section{Future Work}
    Building on the current system, future work should aim for several things. In terms of security and access control, the 
    groundwork was laid for role-based authentication. However, further exploration is needed into\ac{pve}'s native role and 
    permission system. At present, \ac{pve} grants full administrative access to authenticated users, which is acceptable for development environments 
    but poses significant security concerns in production. Currently, role-based access control is enforced solely within the web 
    application layer, meaning that users such as students and teachers interact with the underlying\ac{pve} infrastructure using elevated 
    privileges. A more secure and fine-grained authorization model—where roles like "student" and "teacher" are mapped to 
    corresponding restricted scopes within\ac{pve} itself—remains a critical area for future development. Implementing this would 
    help enforce the principle of least privilege and provide a more robust separation of concerns between the application logic and 
    infrastructure permissions.

    It would also be desirable to expand on the assignment definition model to introduce versioning, as well as introducing 
    capabilities to create more complex grading of exercises including but not limited to adding exercise sub-lines and 
    support for boolean expressions in lines and sublines. 

    Additionally improving on the\ac{ui} using more modern frontend frameworks and shifting to Client Side Rendering could allow 
    for a better UX with new features such as real time feedback and progress tracking of interactions with services external 
    to the web application.

    Developing more, but also introducing more complex modules, capable of chaining commands, and supporting a much wider range 
    of devices is key in making sure this system can go overcome the limitations of its existing counterparts, such as Cisco's 
    Packet Tracer, by supporting more complex, multi-step, assessment of protocols and configurations not available to them.

    Furthermore, exploring the capabilities and limitations of the system in a multi-node cluster environment presents a 
    another direction for future work. While\ac{pve} offers robust support for clustering and centralized management of multiple 
    nodes, it does not natively include mechanisms for automated load balancing across those nodes. In the current implementation,\ac{vm} 
    placement decisions must be made manually, which can lead to inefficient resource utilization, especially under dynamic 
    or heavy workloads. Developing an integrated load balancing feature could significantly enhance scalability, fault tolerance, 
    and system responsiveness. This would involve monitoring node resource usage in real-time and dynamically provisioning or migrating 
    ac{vm}s to balance\ac{cpu} and memory loads more effectively across the cluster.

    Another promising direction for future work lies in providing optional containerized environments for student exercises. While 
    this was not pursued in the current implementation due to the lack of time to perform a thorough risk analysis, the potential 
    benefits justify further investigation. Containerization could reduce resource usage and startup times for exercises that do 
    not require full virtualization, such as those relying solely on\ac{iou} and\ac{vpcs} as well as others that do not rely on\ac{kvm}. 
    
    Simultaneously, as the system grows, it will be vital to integrate unit and system tests to ensure reliability and 
    maintainability as the platform scales, to ensure the continued good functioning of older parts.

    In conclusion, this system represents a promising step toward a fully automated and scalable platform for practical 
    networking education. While there is still much work to be done before it can be fully integrated in institutions and courses, 
    the progress made so far provides a strong basis for further growth and refinement.


% Write text in here
% Use \subsection and \subsubsection to organize text

